\newpage

\thispagestyle{empty}

\begin{center}

{\bf \Huge Resumen}

  \end{center}
\vspace{1cm}

 Con el gran desarrollo que está aconteciendo en la computación cuántica, tanto en términos de la cantidad de qubits en los sistemas cuánticos como en la fiabilidad de estos, nos acercamos cada vez más a poder hacer un uso real de sus algoritmos. Sin embargo, esto nos genera un nuevo problema o, más bien, unas nuevas inquietudes sobre cómo vamos a programarlos, dado que se vuelven cada vez más complejos, y también cómo probar su corrección.\newline 
 
 Aquí es donde presentamos el \textit{testing} metamórfico como una de las posibilidades que tenemos para alcanzar nuestro objetivo. A lo largo de este documento, introduciremos el estudio de propiedades metamórficas para estos algoritmos, así como las implementaciones y el uso del \textit{testing} metamórfico para la detección de fallos.

\vspace{2cm}



\textbf{Palabras clave:} Computación cuántica, qiskit, propiedades metamórficas
   
   
   