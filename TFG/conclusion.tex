\cleardoublepage

\chapter{Conclusiones y trabajo futuro}
\label{Cap5:Conclusion}
 Una vez ya presentadas las MR, así como la aplicación de MT en nuestras implementaciones de algoritmos cuánticos, siendo este el objetivo principal del trabajo, vamos a pasar a analizar todo este largo recorrido y que resultados hemos podido obtener del mismo, así como que posibilidades hemos dejado abiertas a lo largo del texto.

 
\section{Conclusiones}
\label{Sec5.1:Conclusion}

Lo primero que me gustaría destacar de este trabajo es la cantidad de materia que se ha tenido que revisar y comprender para poder alcanzar el objetivo del mismo. Esto ha sido altamente enriquecedor para la comprensión desde un punto de vista más avanzado. Si bien es cierto que todo este aprendizaje ha consumido gran parte del tiempo que se ha dedicado a la realización del TFG, esto se refleja directamente sobre la proporción de páginas que se dedican a esta preparación, que incluyen el  \hyperref[Cap2:Antecedentes]{capítulo 2 de antecedentes} y el \hyperref[Cap3:Algoritmos]{capítulo 3 de algoritmos cuánticos}.\newline

Desde mi punto de vista, lo más importante e interesante de este camino de preparación ha sido comprender la relación directa que existe entre los postulados cuánticos y todos los elementos de la programación cuántica. Se puede establecer una relación uno a uno, lo cual que he intentado expresar y guiar al lector a través de la \hyperref[Sec2.3:Qiskit]{introducción a la programación y Qiskit}.  Además, ha sido importante entender que no todo lo relacionado con lo cuántico es estrictamente probabilístico, sino que tiene sus fundamentos deterministas que se transforman en probabilidades cuando el sistema es observado. \newline

Una vez sentadas estas bases, nos hemos dedicado a estudiar las propiedades metamórficas y como estas MR nos pueden ayudar a encontrar fallos en nuestros algoritmos. La conservación de estas propiedades intrínsecas del algoritmo nos ayuda con el \textit{testing} de programas cuánticos, independientemente de su complejidad y dado que su estructura viene determinada por la \hyperref[Def:MT]{definición de MT}, su implementación resulta sencilla.\newline

Por último, me gustaría destacar cómo se aprecia la necesidad y el respaldo que brindan las matemáticas para cualquier ciencia. La base principal de la creación y verificación formal de los algoritmos cuánticos es puramente matemática, al igual que la obtención de las MR de los mismos y la prueba de su validez.\newline

Me resultó muy interesante el desarrollo realizado para la creación del \hyperref[Sec3.2:Deutsch]{algoritmo de Deutsch} y cómo se puede observar que los algoritmos no salen de la nada, sino que se basan en conocimientos más abstractos como la aplicación de puertas y la prueba matemática de que estos realmente logran lo que queremos. En caso de que no se consiga el resultado deseado, el análisis que hemos realizado puede proporcionarnos una idea para el siguiente paso hacia nuestro objetivo. Lo he incluido en este texto para mostrar al lector el camino realizado desde el problema inicial hasta el algoritmo final.

\section{Trabajo futuro}
\label{Sec5.2:Futuro}

A lo largo de este texto se han ido dejando puertas abiertas hacia posibilidades de estudio, como las recogidas en los retos del \hyperref[Sec2.4:Metamorfico]{\textit{testing} metamórfico}. Estos retos, que intentan guiar a los investigadores hacia nuevas oportunidades o caminos que se deberían completar para una mejor comprensión del MT, se pueden encontrar en el artículo \textit{Metamorphic testing: A new approach for generating next test cases}\cite{AR:MTmain:2008} con una mayor profundidad y número de retos, además este artículo es el que hemos utilizado para el estudio de los conceptos alrededor del \textit{testing} metamórfico.\newline

En cuanto a los avances que se han ido realizando en el campo de la computación cuántica y el MT desde que se comenzó este trabajo, podemos destacar el artículo publicado sobre la corrección de las implementaciones de Shor con \textit{testing} metamórfico\cite{metamorphicShor:2022}, donde se observa los avances y el potencial que tiene el MT sobre algoritmos más complejos de los presentados en este texto. A su vez, se puede observar la falta de capacidad para realizar todas estas pruebas con los sistemas cuánticos actuales y como se va a necesitar de una mejora en el potencial de estos sistemas para poder llevar este $testing$ a una aplicación más completa sobre algoritmos cuánticos más complejos. Además, este año se ha publicado otro articulo que pone a prueba Qiskit, con \textit{testing} metamórfico\cite{AR:QiskitMT:2023} encontrando fallos en el mismo.\newline

\newpage
Respecto a la continuación directa sobre el objetivo de este trabajo, se podrían utilizar distintas aproximaciones:

\begin{itemize}
    \item Continuar con estudios de MR para otros algoritmos, como puede ser QFT o QPE, para el cual se podría utilizar la misma técnica que se utilizó al obtener la \hyperref[RIII:BV]{Regla III del algoritmo de BV}, ya que el QPE utiliza $QFT^{-1}$. Además de los algoritmos de Grover o Shor.

    \item Estudio de otros tipos de \textit{testing} como pruebas de mutación para relacionarlas con MT y poder aplicar ambas a la vez, como se realiza en el artículo \textit{Metamorphic testing of oracle quantum programs} \cite{metamorphicAdd:2022}.
\end{itemize}