\cleardoublepage

\chapter{Introducción}
\label{Cap1:Intro}
La memoria y el trabajo presentado a continuación nos mostrarán el recorrido desde los principio básicos de la mecánica cuántica y cómo definen directamente la computación cuántica, hasta una de las posibilidades que tenemos para hacer \textit{testing} sobre estos programas. Antes de continuar con la metodología, veamos donde comenzó todo. \newline

La mecánica cuántica comenzó a desarrollarse en los años 20, pero no sería hasta los años 80 cuando se empezó a plantear la posibilidad de aplicar esta teoría a la computación\cite{B:QuantumScientist:2008}\footnote{\url{https://en.wikipedia.org/wiki/Quantum_mechanics\#History}}. Paul Benioff presentó en 1980 la máquina de Turing cuántica, que utilizaba la teoría cuántica para describir un ordenador simplificado. En 1984, dicha teoría se utilizó en protocolos criptográficos de la mano de Charlos Bennett y Gilles Brassard. \newline

A partir de entonces, empezaron a surgir nuevos algoritmos, principalmente para resolver el problema de oráculo. Entre ellos,  se encuentran los algoritmos de Deutsch, Deutsch-Jozsa, Bernstein-Vazirani y Simon. Aquí ya se puede observar la mejora en eficiencia en comparación al ordenador clásico, como Richard Feynman conjeturó en 1982\cite{AR:Feynman:1982}. Se podría decir que lo que despertó el interés del resto de la comunidad en la importancia que podía tener la computación cuántica ocurrió gracias a Peter Shor en 1994, con sus algoritmos que podrían permitir romper las claves de encriptación RSA y Diffie-Hellman, que aún se utilizan en la actualidad. Desde entonces, la inversión y el interés en el estudio de este campo han ido creciendo, y en 1998 se alcanzó el hito de construir el primero ordenador de dos qubits, lo que hizo factible esta posibilidad.\newline 

En la actualidad, IBM tiene el ordenador con mayor número de qubits, con 433 qubits, el ibm\_seattle presentado a finales de 2022, y están trabajando para presentar el siguiente sistema cuántico con 1121 qubits a finales de este mismo año\footnote{\url{https://www.ibm.com/quantum/roadmap}}. Es importante destacar que el sistema anterior solo tenía 127 qubits, ¡pero este se presentó hace solo 2 años, en 2021! \newline

Con toda esta evolución que está ocurriendo respecto al número de qubits y las mejoras que están realizando para alcanzar mayor fiabilidad, se empieza a abrir la puerta al uso real de los algoritmos y programas cuánticos. Sin embargo, no se alcanzará hasta conseguir ordenadores con mejores características tanto en número de qubits como en precisión. Pero, ¿cómo vamos a comprobar la corrección de estos?

\vspace{10pt}

Una de las posibilidades que plantearemos en este trabajo es cómo la unión entre la computación cuántica y el \textit{testing} metamórfico puede ayudarnos a contestar esta pregunta y buscar esa corrección o falta de errores. El \textit{testing} metamórfico es uno de los métodos utilizados para la detección de errores desde su presentación en 1998\cite{Note:MT:1998}, y se basa en el estudio de las propiedades necesarias que pueden obtenerse de los algoritmos.

\section{Metodología}
\label{Sec1.1:Metodologia}

La metodología seguida en este trabajo se puede distinguir en tres fases, las cuales se reflejarán en los siguientes capítulos de esta memoria. El material principal utilizado como base de estudio han sido los libros \textit{Quantum Computation and Quantum Information}, de Michael A. Nielsen y Isaac L. Chuang\cite{B:Nielsen:2002}, y \textit{Quantum Computing for Computer Scientists}, de Noson S. Yanofsky y Mirco A. Mannucci\cite{B:QuantumScientist:2008}.A continuación, vamos a introducir brevemente cada una de estas fases: 

\begin{itemize}
    \item \textbf{Antecedentes}: Esta primera fase de estudio, nos enfocamos en avanzar desde los conocimientos básicos adquiridos en el grado, hasta adquirir una base sólida para poder entender los algoritmos cuánticos, la obtención de las propiedades metamórficas y su aplicación en el \textit{testing}. Además de los textos mencionados anteriormente, que han proporcionado resultados más teóricos, en este capítulo se han establecido los conceptos relacionados con el \textit{testing} y propiedades metamórficas, tomando como referencia el artículo \textit{Metamorphic Testing: A Review of Challenges and Opportunities}\cite{AR:MTmain:2008}.
    
    \item \textbf{Programación cuántica y algoritmos}: Una vez adquirida la base necesaria para comprender los algoritmos y la programación cuántica, se llevan a cabo los primeros pasos de programación, empezando por un algoritmo tan sencillo como la suma y la implementación de la misma utilizando Qiskit. A partir de ahí, se fue avanzando algoritmo por algoritmo, desde los más simples hasta llegar a la transformada cuántica de Fourier y aplicaciones. El estudio de estos algoritmos se presentará en el capítulo \ref{Cap3:Algoritmos}, y todos las implementaciones y pruebas realizadas sobre estos algoritmos, para no sobrecargar este documento, se encuentran en el repositorio personal de GitHub, \url{https://github.com/sinugarc/TFG.git} . Además, para el dibujo de circuitos en la memoria se ha utilizado Qcircuit\cite{AR:QcircT:2004}. 
    
    \item \textbf{Propiedades y \textit{testing} metamórfico}: Ahora que ya hemos logrado entender e incluso programar nuestros algoritmos cuánticos, es hora de abordar el último eslabón de la cadena: el estudio de las propiedades metamórficas en los algoritmos de Deutsch-Jozsa, Bernstein-Vazirani y Simon. Como apoyo para la comprensión y estudio de este fase, hemos utilizado como referencia en el artículo \textit{Metamorphic Testing of Oracle Quantum Programs}\cite{metamorphicAdd:2022}. Además, esta parte del trabajo se ha realizado en colaboración con Rodrigo de la Nuez Moraleda, y todas las implementaciones realizadas para el \textit{testing} de estos algoritmos se pueden encontrar en el repositorio de GitHub que tenemos en común, \url{https://github.com/rodelanu/TFG} .
\end{itemize}

\section{Objetivos}
\label{Sec1.2:Objetivos}

Una vez analizada la metodología de trabajo y los materiales utilizados, vamos a seguir el mismo esquema para analizar los objetivos. Es importante destacar que el objetivo principal de este trabajo es el estudio de las propiedades metamórficas y su aplicación en forma de \textit{testing} a algoritmos cuánticos. Sin embargo, para llegar a este punto, vamos a ir alcanzando una serie de objetivos y adquiriendo competencias secundarias en cada fase del trabajo. 

\begin{itemize}
    \item \textbf{Antecedentes}: Queremos ser capaces de entender las definiciones más básicas, partiendo de los conocimientos matemáticos, especialmente del álgebra lineal. Algunos ya conocidos como que representa una matriz unitaria o hermitiana, así como la relación que existe entre ellas y sus operadores. También se introducirán nuevos conceptos, como el espacio de Hilbert y el producto tensorial, este último nos permitirá generar sistemas más complejos. Además, estudiaremos la parte física de computación cuántica, con sus postulados y su relación con dicha programación. También vamos a  introducir los elementos básicos de dicha programación y comprender el sentido general de las simulaciones y ejecuciones de los programas cuánticos.
    
    \item \textbf{Programación cuántica y algoritmos}: El objetivo principal de este capítulo es claro: queremos ser capaces de crear y analizar programas. Para este fin, nos enfocaremos en entender cómo se han construido los diversos algoritmos y la utilidad que tienen.

    \item \textbf{Propiedades y \textit{testing} metamórfico}: Aquí se desarrollará el objetivo principal del trabajo.
\end{itemize}

\vspace{1cm}