\cleardoublepage

\chapter{Algoritmos cuánticos}
\label{makereference}

El siguiente paso en este camino hacia la unión entre la computación cuántica y el \textit{testing} metamórfico es la creación de algoritmos o programas cuánticos que nos ayuden a resolver problemas propuestos o avanzar. \newline

Todo el desarrollo de estos algoritmos se pueden encontrar en los libros principales (Nielsen//CforInf). Presentaremos a continuación todos los algoritmos finales que nos permiten obtener nuestros objetivos, si bien es cierto, solo vamos a presentar el camino completo de creación del algoritmo de Deutsch-Jozsa. Esto se debe a que para alcanzar nuestro objetivo debemos ir haciendo modificaciones y cálculos sobre nuestros algoritmos hasta dar con la combinación correcta de puertas que nos permita resolverlo.\newline

Esta sección va a seguir prácticamente la misma estructura para cada apartado, Empezará con una presentación, seguida de la exposición del problema a resolver. Entonces nos dispondremos a presentar el algoritmo que lo resuelve, o su creación, y las pruebas realizadas. Es necesario recordar que toda la programación realizada sobre estos algoritmos se encuentra en el repositorio de GitHub \url{https://github.com/sinugarc/TFG.git}

\section{Suma}
 Este primer algoritmo nos va a servir como primera toma de contacto con la programación cuántica, las puertas que podemos utilizar y el uso de \textit{Qiskit}. Veamos cual es el problema a resolver:

\section{DJ}
 Algoritmo de Deutsch-Jozsa \n

\section{BV}
Algoritmo de Berstein- Vazirani \n

\section{Simon}
Algoritmo de Simon \n

\section{QFT}
Algoritmo de la transformada cuántica de Fourier \n

\section{QEP}
Algoritmo cuántico de estimación de fase \n

\section{Grover}
Algoritmo cuántico de busqueda de Grover \n