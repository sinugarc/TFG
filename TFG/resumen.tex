\newpage

\thispagestyle{empty}

\begin{center}

{\bf \Huge Resumen}

  \end{center}
\vspace{1cm}

 Con el gran desarrollo que está aconteciendo en la computación cuántica, cada vez estamos más cerca de poder hacer un uso real de sus algoritmos. Pero esto nos genera un nuevo problema, o más bien, unas nuevas inquietudes sobre cómo vamos a programarlos, ya que cada vez son más complejos, y además cómo vamos a probar su corrección. Aquí, es donde presentamos nuestro estudio de las propiedades metamórficas de estos algoritmos, que nos van a ayudar a introducir el \textit{testing} metamórfico, siendo esta una de las posibilidades que tenemos para alcanzar nuestro objetivo.

\vspace{2cm}



\textbf{Palabras clave:} Computación cuántica, qiskit, propiedades metamórficas
   
   
   