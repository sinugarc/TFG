\newpage

\thispagestyle{empty}

\begin{center}

{\bf \Huge Abstract}

  \end{center}
\vspace{1cm}

The quick development that quantum computing is having in the latest years is opening new chances within the field. IBM presented in 2022 his newest developed processor, \textit{Osprey}, with a 433 qubit structure. They have already set up on their roadmap the possibility to offer his new processor with 1121 qubits by the end of this year. This number will open the research in some prototype software applications within IBM\footnote{\url{https://www.ibm.com/quantum/roadmap}}.\newline

Looking at the progress made in the recent years towards qubit capacity and reliability, is making us get closer to unlock the real use of quantum algorithms, although we may need a few thousand more. But as soon as we think of using this algorithms, how are we going to probe their correctness?\newline

We are going to present through this paper one of the alternatives to probe correctness or the lack of faults. We are going to show the possibility of bringing together quantum computing and metamorphic testing, based in logic properties directly from the algorithm. This option has been used as fault finder since 1998 when it was presented.

\vspace{1cm}

\textbf{Keywords:} Quantum computing, qiskit, metamorphic properties.