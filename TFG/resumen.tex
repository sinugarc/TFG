\newpage

\thispagestyle{empty}

\begin{center}

{\bf \Huge Resumen}

  \end{center}
\vspace{1cm}

 El desarrollo de la computación cuántica esta en el punto de mira debido a los avances que está teniendo estos últimos años. IBM presentó a finales del año pasado un ordenador con 433 qubits y trabajando para presentar el siguiente sistema cuántico con 1121 qubits a finales de este mismo año. Hay que destacar que el anterior sistema solo tenían 127 qubits, ¡pero este se presentó solo hace 2 años, en 2021! \newline

Con toda esta evolución que está ocurriendo a nivel del número de qubits y las mejoras que se van realizando para alcanzar mayor fiabilidad, empieza a abrir la puerta al uso real de los algoritmos y programas cuánticos. Aunque estos no llegarán hasta conseguir ordenadores con mejores características. Pero, ¿cómo vamos a comprobar la corrección de estos? \newline 

Una de las posibilidades que vamos a plantear en este trabajo es como la unión entre la computación cuántica y el \textit{testing} metamórfico puede ayudarnos a contestar esta pregunta y a intentar buscar esa corrección o falta de errores. Donde el \textit{testing} metamórfico es uno de los métodos usados para la búsqueda de errores desde que se presentó en 1998, basado en el estudio de las propiedades lógicas que se pueden obtener de los algoritmos.

\vspace{1cm}



\textbf{Palabras clave:} Computación cuántica, qiskit, propiedades metamórficas
   
   
   