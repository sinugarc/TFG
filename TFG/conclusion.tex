\cleardoublepage

\chapter{Conclusiones y trabajo futuro}
\label{Cap5:Conclusion}
 Una vez que se han presentado las MR, así como la aplicación del MT en nuestros algoritmos cuánticos, que es el objetivo principal del trabajo, pasaremos a analizar todo este largo recorrido y los resultados obtenidos. También exploraremos las posibilidades hemos dejado abiertas a lo largo del texto.

 
\section{Conclusiones}
\label{Sec5.1:Conclusion}

Lo primero que me gustaría destacar de este trabajo es la cantidad de materia que se ha tenido que revisar y comprender para poder alcanzar nuestro objetivo. Esto ha sido altamente enriquecedor para la comprensión desde un punto de vista más avanzado. Si bien es cierto que todo este aprendizaje ha consumido gran parte del tiempo que se ha dedicado a la realización del TFG, esto se refleja directamente sobre la proporción de páginas que se dedican a esta preparación, que incluyen el  \hyperref[Cap2:Antecedentes]{capítulo 2 de antecedentes} y el \hyperref[Cap3:Algoritmos]{capítulo 3 de algoritmos cuánticos}.\newline

Desde mi punto de vista, lo más importante e interesante de este proceso de preparación ha sido comprender la relación directa que existe entre los postulados cuánticos y todos los elementos de la programación cuántica. Se puede establecer una correspondencia uno a uno, que he intentado expresar y guiar al lector a través de la \hyperref[Sec2.3:Qiskit]{introducción a la programación y Qiskit}.  Además, ha sido importante entender que no todo lo relacionado con lo cuántico es estrictamente probabilístico, sino que tiene sus fundamentos deterministas que se transforman en probabilidades cuando el sistema es observado. \newline

Una vez establecidas estas bases, nos hemos dedicado al estudio las propiedades metamórficas y como estas MR pueden ayudarnow a encontrar fallos en nuestros algoritmos. La conservación de estas propiedades intrínsecas del algoritmo nos ayuda con el \textit{testing} de programas cuánticos, independientemente de su complejidad. Además, dado que su estructura está determinada por la \hyperref[Def:MT]{definición de MT}, su implementación resulta sencilla.\newline

Por último, me gustaría destacar la necesidad y el respaldo que las matemáticas brindan a todas las ciencias. En el caso de la programación cuántica, la base principal para la creación y verificación determinista de los algoritmos es especialmente matemática. De igual manera, la obtención de las MR y la prueba de su validez se fundamentan en principios matemáticos.\newline

Me resultó muy interesante el desarrollo realizado para la creación del \hyperref[Sec3.2:Deutsch]{algoritmo de Deutsch}.  Es evidente que los algoritmos no surgen de la nada, sino que se basan en conocimientos más abstractos, como la aplicación de puertas cuánticas y la prueba matemática de su efectividad. En caso de que no lograr el resultado deseado, el análisis realizado nos puede proporcionar ideas para dar el siguiente paso hacia nuestro objetivo. Lo he incluido en este texto para mostrar al lector el camino realizado desde el problema inicial hasta el algoritmo final.

\section{Trabajo futuro}
\label{Sec5.2:Futuro}

A lo largo de este texto se han ido dejando puertas abiertas hacia posibilidades de estudio, como las recogidas en los retos del \hyperref[Sec2.4:Metamorfico]{\textit{testing} metamórfico}. Estos retos, que intentan guiar a los investigadores hacia nuevas oportunidades o caminos que se deberían completar para una mejor comprensión del MT, se pueden encontrar en el artículo \textit{Metamorphic testing: A new approach for generating next test cases}\cite{AR:MTmain:2008} con un mayor número de retos y profundidad, además este artículo es el que hemos utilizado para el estudio de los conceptos referentes al \textit{testing} metamórfico.\newline

En cuanto a los avances que se han ido realizando en el campo de la computación cuántica y el MT desde que se comenzó este trabajo, podemos destacar el artículo publicado sobre la corrección de las implementaciones de Shor con \textit{testing} metamórfico\cite{metamorphicShor:2022}, donde se observa los avances y el potencial que tiene el MT sobre algoritmos más complejos de los presentados en este texto. A su vez, se puede observar la falta de capacidad para realizar todas estas pruebas con los sistemas cuánticos actuales y como se va a necesitar de una mejora en el potencial de estos sistemas para poder llevar este $testing$ de manera efectiva sobre algoritmos cuánticos más complejos. Además, este año se ha publicado otro articulo que pone a prueba Qiskit con \textit{testing} metamórfico\cite{AR:QiskitMT:2023} encontrando fallos en el mismo.\newline

\newpage
Respecto a la continuación directa sobre el objetivo de este trabajo, se podrían utilizar distintas aproximaciones:

\begin{itemize}
    \item Profundizar en otros algoritmos como la transformada cuántica de Fourier, en adelante QFT, y sus aplicaciones, como la estimación de fase, QPE, que se comenzaron a estudiar al igual que el algoritmo de Grover. Sin embargo, no tuvimos tiempo para la total comprensión y estudio de reglas metamórficas. Una posible idea para QFT sería utilizar la misma técnica que se usó para obtener la \hyperref[RIII:BV]{Regla III del algoritmo de BV}, ya que el QPE utiliza $QFT^{-1}$.

    \item Estudio de otros tipos de \textit{testing} como pruebas de mutación para relacionarlas con MT y poder aplicar ambas a la vez. Esta técnica se  utiliza en el artículo \textit{Metamorphic testing of oracle quantum programs} \cite{metamorphicAdd:2022}.
\end{itemize}

Para finalizar, solo me faltaría destacar que los algoritmos cuánticos siguen creciendo conforme pasan los días, cada vez aplicados a más ramas de las matemáticas. Una de las mayores colecciones que hemos obtenido, se encuentra en \url{https://quantumalgorithmzoo.org/} de donde se podrían obtener otros algoritmos que pudiéramos poner en estudio.