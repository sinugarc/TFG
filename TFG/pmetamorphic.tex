\cleardoublepage

\chapter{Propiedades metamórficas}
\label{Cap4:PMetamorficas}

En este último capítulo antes de la conclusión queremos presentar al lector con el objetivo principal de este trabajo. Queremos finalmente unir las propiedades metamórficas con los algoritmos del capítulo \ref{Cap3:Algoritmos}, de los cuales se estudiarán únicamente los tres últimos. A continuación, presentaremos cuales son las propiedades obtenidas, así como de donde proceden, para luego pasar a la programación y los circuitos que hemos preparado para poder realizar las pruebas necesarias.

\section{Deutsch-Jozsa}
\label{Sec4.1:DJ}
En esta sección vamos a estudiar qué reglas hemos obtenido del \hyperref[Sec3.3:Deutsch-Jozsa]{algoritmo de Deutsch-Jozsa} y cómo las hemos implementado en nuestro \href{https://github.com/rodelanu/TFG/blob/main/1_Deutsch_Jozsa_Rules.ipynb}{repositorio}. Recordamos brevemente el problema y el algoritmo obtenido:\newline

\textbf{\hyperref[P:DJ]{Problema}}: Dada una función $f:\{0,1\}^{n} \rightarrow\{0,1\}$ balanceada o constante, la cual no podemos observar su definición. Queremos determinar si esta función es constante o balanceada.\newline

\textbf{\hyperref[A:DJ]{Algoritmo de Deutsch-Jozsa}}:

 \vspace{10pt}

 \begin{center}$\Qcircuit @C=1.5em @R=1em {
 \lstick{|\mathbf{0}\rangle}& \qw {/^{n}} & \gate{H^{\otimes n}} & \qw {/^{n}} & \multigate{1}{U_{f}} & \qw {/^{n}} & \gate{H^{\otimes n}} & \qw {/^{n}} & \meter & \qw \\ \lstick{|1\rangle} & \qw & \gate{H} & \qw &\ghost{U_{f}} & \qw & \qw & \qw  & \qw & \qw}$ \end{center}

\vspace{12pt}

\textbf{Resultado}: Se obtiene $|\mathbf{0}\rangle$ si $f$ es constante y cualquier otro resultado si $f$ es balanceada.

\vspace{12pt}

\textbf{Observación}: No hay que olvidar la importancia de tener como hipótesis que $f$ es constante o balanceada. Pero en este caso, debido a que esta va a formar parte de nuestro \textit{source input}, podemos asegurar esta propiedad. \newline

Suponemos $P$ la implementación del algoritmo de Deutsch-Jozsa a analizar, veamos qué reglas hemos obtenido y su implementación: \newline


\textbf{Regla I}: Sea $f:\{0,1\}^{n} \rightarrow\{0,1\}$ la función a analizar definida en problema y sea $g : \{0,1\}^{n} \rightarrow \{0,1\}^{n}$ un automorfismo. Entonces, $P(f)=|\mathbf{0}\rangle$ si y solo si $P(f\circ g)=|\mathbf{0}\rangle$.\newline

Al aplicar el automorfismo $g$, $f \circ g$ no varía el número de 1 y 0 de la imagen y por lo tanto no cambia la característica de la función, por lo que es directo decir que la MR es correcta.\newline

\textbf{Implementación}: Comparamos los resultados entre la simulación $P(f)$ y $P(f\circ g)$. En este caso hemos decidido utilizar un automorfismo particular a la hora de realizar las pruebas. Sea $g: \{0,1\}^{n} \rightarrow \{0,1\}^{n}$ talque aplica la puerta $X$ a cada componente, es decir, $g(\mathbf{x})=\mathbf{1}-\mathbf{x}$ que es el complementario binario de $\mathbf{x}$.\newline



\textbf{Regla II}: Sea $f:\{0,1\}^{n} \rightarrow\{0,1\}$ la función a analizar definida en problema y sea $h:\{0,1\}^{n} \rightarrow\{0,1\}$ tal que $h(x) = 1 - f(x)$. Entonces, $P(f)=|\mathbf{0}\rangle$ si y solo si $P(h)=|\mathbf{0}\rangle$. \newline

Al igual que con la regla I, no se varía la condición de $f$ ya que si esta es balanceada mantiene el número de 0 y 1, ya que tienen el mismo. Y si fuera constante, simplemente cambiaría el valor de la constante.\newline

\textbf{Implementación}: Este caso es análogo al anterior, simplemente cambiamos la condición del qubit ancilla: le aplicamos una puerta $X$ tras el oráculo.\newline

Es cierto que en el repositorio\footnote{\url{https://github.com/rodelanu/TFG/blob/main/1_Deutsch_Jozsa_Rules.ipynb}}, así como en la programación realizada de acuerdo a MT, consideramos que el resultado de $P$ es un valor binario, esto lo obtenemos haciendo el máximo de los bit obtenidos. Cabe recordar que si una función es balanceada el resultado solo implica que no es $|\mathbf{0}\rangle$, es decir en nuestros bits tendremos al menos un uno, que nos determinará el máximo a 1. Por lo que podríamos concluir que a nivel de programación en MT $P \rightarrow 0$ si es constante y $P \rightarrow 1$ si es balanceada. Las reglas expresadas de forma matemáticamente exacta de acorde en al algoritmo en este documento son equivalentes a la expresadas en el repositorio, tras la transformación explicada en este mismo párrafo. 

\section{Bernstein-Vazirani}
\label{Sec4.2:BV}
El algoritmo de Bernstein-Vazirani fue el primero presentado como ejemplo al introducir la \hyperref[Sec2.3:Qiskit]{programación cuántica} en el \hyperref[Cap2:Antecedentes]{capítulo 2}. Recapitulemos sobre lo estudiado del \hyperref[Sec3.4:BV]{algoritmo de BV}.\newline

\textbf{Problema}: Sea $f:\{0,1\}^{n} \rightarrow \{0,1\}$, sea $\mathbf{s}=s_{0}s_{1}...s_{n-1} \in \{0,1\}^{n}$, por hipótesis sabemos que $ f(\mathbf{x})=(\mathbf{s}\cdot \mathbf{x})\: \text{mod}\:2 = s_{0}x_{0} \oplus s_{1}x_{1} \oplus ... \oplus s_{n-1}x_{n-1}$. Queremos obtener la cadena $\mathbf{s}$.\newline

\textbf{Algoritmo de Bernstein-Vazirani}:

 \vspace{10pt}

 \begin{center}$\Qcircuit @C=1.5em @R=1em {
 \lstick{|\mathbf{0}\rangle}& \qw {/^{n}} & \gate{H^{\otimes n}} & \qw  & \qw {/^{n}} & \multigate{1}{U_{f}} & \qw {/^{n}} & \gate{H^{\otimes n}} & \qw {/^{n}} & \meter & \qw \\ \lstick{|0\rangle} & \qw & \gate{H} & \gate{Z} & \qw &\ghost{U_{f}} & \qw & \qw & \qw  & \qw & \qw}$ \end{center}

 \vspace{30pt}

 \textbf{Resultado}: Obtenemos la cadena $|\mathbf{s}\rangle$ en la medición.\newline

 Si suponemos que P es una implementación del algoritmo de BV, vamos a estudiar la reglas obtenidas: \newline

 \textbf{Regla I}\label{R:BV:1}: Sea $\mathbf{s}$ una cadena binaria de longitud $n$ y sea $\mathbf{\Bar{s}}$ su complementario binario. Entonces $P(\mathbf{s})\oplus P(\mathbf{\Bar{s}})=|\mathbf{1}\rangle$ de longitud $n$.\newline

 Esta regla se ha obtenido como un caso particular de la regla general que se presentará a continuación. Ahora bien, ¿Es trascendente el particularizar una regla ya obtenida? Hagamos un pequeño inciso para aclarar este tema.\newline

 La respuesta a esta pregunta es sí, ya que aunque una regla más general cubre muchos más casos, esto no nos asegura que encontrar un fallo sea más fácil. Sin embargo, puede ser incluso más complejo programar un regla general. Es cierto que este caso es muy simple y la comprobación para es directa, ya que queremos obtener el ket $|\mathbf{1}\rangle$. \newline

 El ejemplo básico que se usa para explicar esta diferencia es el siguiente \cite{AR:MTmain:2008}:\newline

Suponemos que tenemos una implementación $P$ con una MR tal que $f(k\times x)=k\times f(x)$, donde $k \in \mathbb{Z} \setminus \{0\}$. De esta manera se puede extrapolar fácilmente de un \textit{source input} a infinitos casos al fijar un $x$ y variar $k \in \mathbb{N}$. Pero en realidad, sigue dejando muchísimos casos sin comprobar y tiene cierta dificultad para confirmar todos los \textit{follow-up inputs} en $P$, debido a la cantidad de comprobaciones a realizar para un mismo \textit{source input}. Por otra parte, si tomamos $k=-1$ tenemos otra MR, que aún siendo más débil que la anterior es más fácil de verificar. Y evidentemente, cualquier fallo que se dé con $f(-x)=-f(x)$ es tan efectivo como uno encontrado con la regla general.\newline

Ya nos podemos centrar en nuestra regla I. Para programar este test, vamos a ejecutar por separado $P$ para cada cadena $\mathbf{s}$ y $\mathbf{\Bar{s}}$ y realizaremos la \hyperref[Sec3.1:Suma]{suma bit a bit} de los resultados antes de realizar la medición. La razón por la que esperamos el ket $|\mathbf{1}\rangle$ se debe a que si uno es el complementario binario del otro, su suma binaria es 1 en cada posición.\newline

 El circuito utilizado es idéntico a la Figura \ref{Fig:CircuitoBVReglaII}, simplemente $\mathbf{s}1$ cumplirá que $\mathbf{s}\oplus\mathbf{s}1=|\mathbf{1}\rangle$.

\begin{figure}[H]
    \centering
    \includegraphics[width=\textwidth]{TFG/imagenes/BVReglaII.png}
    \caption{Circuito para la regla II de BV}
    \label{Fig:CircuitoBVReglaII}
 \end{figure}


\textbf{Regla II}: Sean $\mathbf{s}$ y $\mathbf{s}'$ cadenas binarias de longitud $n$. Entonces $P(\mathbf{s})\oplus P(\mathbf{s}')=\mathbf{s} \oplus \mathbf{s}'$.\newline

 Se puede entender que esta es la generalización de la \hyperref[R:BV:1]{regla I}, si bien es cierto podría incluso generalizarse para 2 cadenas que no tengan que tener la misma longitud o incluso se podría interpretar como $P(f_{\mathbf{s}
}\oplus f_{\mathbf{s}'})=P(f_{\mathbf{s}})\oplus P(f_{\mathbf{s}'})$, relacionando así dos \textit{output} del \textit{source input} con uno del \textit{follow-up output}.\newline

 Las reglas anteriores han sido obtenidas de forma directa por la naturaleza del problema, donde si el resultado es la cadena con la que hemos generado el oráculo, entonces la suma binaria se debe conservar tanto en el \textit{source input} como en el \textit{output}.\newline

 El circuito utilizado se muestra en la Figura \ref{Fig:CircuitoBVReglaII}. Es el circuito creado para las cadenas $\mathbf{s}=011$ y $\mathbf{s}1=001$, las cuales se generan de manera aleatoria una vez determinada una longitud $n$ que se pasa como parámetro a la función.\newline

 \textbf{Regla III}\label{RIII:BV}: Sean $\mathbf{s}$ y $\mathbf{s}'$ cadenas binarias de longitud $n$. Si entendemos la composición como la concatenación de sus oráculos, entonces $P(f(\mathbf{s}) \circ f(\mathbf{s}'))= P(f_{\mathbf{s}}) \oplus P(f_{\mathbf{s}'})$.\newline

 La primera vez que se pensó en esta regla fue directamente desde las ecuaciones del algoritmo, en particular, la ecuación \ref{eq:BV:phi2}. Tras aplicar la primera $f$, en este caso $f_{\mathbf{s}}$, obtendríamos:

 \begin{equation} 
    \mathbf{|\varphi_{2}\rangle} =\left[ \dfrac{\sum_{\mathbf{x} \in \{0,1\}^{n}}(-1)^{f_{\mathbf{s}}(\mathbf{x})}|\mathbf{x}\rangle}{\sqrt{2^{n}}}\right] \left[ \dfrac{|0\rangle - |1\rangle}{\sqrt{2}}\right]\end{equation}\newline

Veamos como quedaría el circuito con la concatenación, que incluye este estado:

\vspace{20pt}

 \begin{center}$\Qcircuit @C=1.5em @R=1em {
 \lstick{|\mathbf{0}\rangle}& \qw {/^{n}} & \gate{H^{\otimes n}} & \qw  & \qw {/^{n}} & \multigate{1}{U_{f_{\mathbf{s}}}} & \qw {/^{n}} & \multigate{1}{U_{f_{\mathbf{s}'}}} & \qw {/^{n}} & \gate{H^{\otimes n}} & \qw {/^{n}} & \meter & \qw \\ \lstick{|0\rangle} & \dstick{\begin{matrix} \Uparrow \\ |\varphi_{0}\rangle \end{matrix}} \qw & \gate{H} & \gate{Z} & \dstick{\begin{matrix} \Uparrow \\ |\varphi_{1}\rangle \end{matrix}} \qw &\ghost{U_{f}} & \dstick{\begin{matrix} \Uparrow \\ |\varphi_{2}\rangle \end{matrix}} \qw &\ghost{U_{f_{\mathbf{s}'}}} & \dstick{\begin{matrix} \Uparrow \\ |\varphi_{3}\rangle \end{matrix}} \qw & \qw & \dstick{\begin{matrix} \Uparrow \\ |\varphi_{4}\rangle \end{matrix}} \qw  & \qw & \qw}$ \end{center}

 \newpage

 Desde aquí podemos desarrollar $|\varphi_{3}\rangle$:

 \begin{equation}
    \begin{split}
     \mathbf{|\varphi_{3}\rangle} &= \left[ \dfrac{\sum_{\mathbf{x} \in \{0,1\}^{n}}(-1)^{f_{\mathbf{s}}(\mathbf{x})}(-1)^{f_{\mathbf{s}'}(\mathbf{x})}|\mathbf{x}\rangle}{\sqrt{2^{n}}}\right] \left[ \dfrac{|0\rangle - |1\rangle}{\sqrt{2}}\right] \\ &= \left[ \dfrac{\sum_{\mathbf{x} \in \{0,1\}^{n}}(-1)^{f_{\mathbf{s}}(\mathbf{x})\oplus f_{\mathbf{s}'}(\mathbf{x})}|\mathbf{x}\rangle}{\sqrt{2^{n}}}\right] \left[ \dfrac{|0\rangle - |1\rangle}{\sqrt{2}}\right] \\ &= \left[ \dfrac{\sum_{\mathbf{x} \in \{0,1\}^{n}}(-1)^{\langle\mathbf{s},\mathbf{x}\rangle\oplus \langle\mathbf{s}',\mathbf{x}\rangle}|\mathbf{x}\rangle}{\sqrt{2^{n}}}\right] \left[ \dfrac{|0\rangle - |1\rangle}{\sqrt{2}}\right] \\ &= \left[ \dfrac{\sum_{\mathbf{x} \in \{0,1\}^{n}}(-1)^{\langle\mathbf{s}\oplus\mathbf{s}',\mathbf{x}\rangle}|\mathbf{x}\rangle}{\sqrt{2^{n}}}\right] \left[ \dfrac{|0\rangle - |1\rangle}{\sqrt{2}}\right] \\ &= \left[ \dfrac{\sum_{\mathbf{x} \in \{0,1\}^{n}}(-1)^{f_{\mathbf{s}\oplus\mathbf{s}'}(\mathbf{x})}|\mathbf{x}\rangle}{\sqrt{2^{n}}}\right] \left[ \dfrac{|0\rangle - |1\rangle}{\sqrt{2}}\right]
     \end{split}
 \end{equation}\newline

 Con este desarrollo obtenemos directamente la regla III, ya que llegamos exactamente a la misma ecuación desde el circuito definido para $f_{\mathbf{s}\oplus\mathbf{s}'}$, donde $|\varphi_{3}\rangle=|\varphi_{2}'\rangle$ .

 \vspace{20pt}

 \begin{center}$\Qcircuit @C=1.5em @R=1em {
 \lstick{|\mathbf{0}\rangle}& \qw {/^{n}} & \gate{H^{\otimes n}} & \qw  & \qw {/^{n}} & \multigate{1}{U_{f_{\mathbf{s}\oplus\mathbf{s}'}}} & \qw {/^{n}} & \gate{H^{\otimes n}} & \qw {/^{n}} & \meter & \qw \\ \lstick{|0\rangle} & \dstick{\begin{matrix} \Uparrow \\ |\varphi_{0}'\rangle \end{matrix}} \qw & \gate{H} & \gate{Z} & \dstick{\begin{matrix} \Uparrow \\ |\varphi_{1}'\rangle \end{matrix}} \qw &\ghost{U_{f_{\mathbf{s}\oplus\mathbf{s}'}}} & \dstick{\begin{matrix} \Uparrow \\ |\varphi_{2}'\rangle \end{matrix}} \qw & \qw & \dstick{\begin{matrix} \Uparrow \\ |\varphi_{3}'\rangle \end{matrix}} \qw  & \qw & \qw}$ \end{center}

 \vspace{50pt} 

 Para la implementación de esta regla, hemos querido reducir la complejidad de las comparaciones y así evitar un uso excesivo de qubits. Hay que recordar que las matrices que usan los simuladores crecen de manera exponencial, ya que la base de $n$ qubits tiene $2^{n}$ elementos. Además en IBM, el máximo sistema cuántico de acceso libre tiene sólo 7 qubits, por lo que quizás para esta regla no tuviéramos problemas, pero para la regla I y II, seguramente no podríamos ejecutarlos.\newline

 Por lo que implementamos $P(f(\mathbf{s}) \circ f(\mathbf{s}'))$, ya que la suma por separado sería análoga a las reglas anteriores. De esta manera el circuito que utilizaremos a la hora de realizar MT será, \newline

 \begin{figure}[H]
    \centering
    \includegraphics[width=\textwidth]{TFG/imagenes/BVRegla3.png}
    \caption{Circuito para la regla III de BV}
    \label{Fig:CircuitoBVReglaIII}
 \end{figure}

 Los circuitos y pruebas realizadas se puede encontrar en el archivo de BV \footnote{\url{https://github.com/rodelanu/TFG/blob/main/2_Bernstein_Vazirani_Rules.ipynb}} del \href{https://github.com/rodelanu/TFG}{repositorio común}.

 
\section{Simon}
\label{Sec4.3:Simon}

Por último vamos a terminar de estudiar el algoritmo de Simon y como vamos a obtener las reglas metamórficas. En este caso no van a ser una aplicación tan directa como los algoritmos anteriores, debido a que el algoritmo de Simon acaba con computación clásica. Por lo que para la obtención de MR, nos vamos a centrar en ese punto intermedio entre el uso de la computación cuántica y el paso a lo clásico para solucionar los sistemas de ecuaciones que vimos en la presentación del algoritmo en la sección \ref{Sec3.5:Simon} \newline

\textbf{Problema}: Dada una función $f:\{0,1\}^{n} \rightarrow\{0,1\}^{n}$ con periodo $\mathbf{c}=c_{0}c_{1}...c_{n-1}$. Queremos determinar cual es el periodo de $f$ teniendo en cuenta que no conocemos su definición.\newline

\textbf{Algoritmo de Simon}, parte cuántica:

 \vspace{3pt}

 \begin{center}$\Qcircuit @C=1.5em @R=1em {
 \lstick{|\mathbf{0}\rangle}& \qw {/^{n}} & \gate{H^{\otimes n}} & \qw {/^{n}} & \multigate{1}{U_{f}} & \qw {/^{n}} & \gate{H^{\otimes n}} & \qw {/^{n}} & \meter & \qw \\ \lstick{|\mathbf{0}\rangle} & \qw & \qw {/^{n}} & \qw &\ghost{U_{f}} & \qw & \qw {/^{n}} & \qw  & \qw & \qw}$ \end{center}

 \vspace{30pt}

 \textbf{Resultados}: Obtención de $n$ cadenas binarias tal que si $\mathbf{z}$ es una de estas cadenas, entonces $\langle \mathbf{z},\mathbf{c}\rangle = 0$.\newline

 Supongamos ahora que tenemos $P$ implementación del algoritmo de Simon, veamos que MR hemos podido obtener para este algoritmo.\newline

 \textbf{Regla I}: Al efectuar la suma bit a bit con los posibles resultados del programa que resuelve el algoritmo de Simon, obtenemos de nuevo el conjunto inicial. Es decir, forma un grupo con la suma bit a bit. \newline

 Vamos a comprobar que es cierto que forma un grupo con esa operación interna, de esta manera ya habremos fundamentado de forma teórica la base de esta regla. Sea $S_{\mathbf{c}}$ el conjunto de las $\mathbf{z}$ cadenas que se puede obtener con la implementación $P$ del algoritmo de Simon para una cierta cadena $\mathbf{c}$ y $\oplus$ la suma bit a bit:

 \begin{itemize}
     \item $S_{\mathbf{c}} \neq \varnothing$: Veamos que $\mathbf{0} \in S_{\mathbf{c}}$, $\forall\: \mathbf{c} \in \{0,1\}^{n}$. Sea $\mathbf{c} \in \{0,1\}^{n}$, $\langle\mathbf{0},\mathbf{c}\rangle=0 \Rightarrow \mathbf{0} \in S_{\mathbf{c}}$

     \item $\oplus$ \textbf{es una operación interna}: Sean $\mathbf{c} \in \{0,1\}^{n}$ y $\mathbf{z}$, $\mathbf{z}' \in S_{\mathbf{c}}$. Tenemos que comprobar que $\mathbf{z} \oplus\mathbf{z}' \in S_{\mathbf{c}}$:

     \begin{equation}
         \langle\mathbf{z} \oplus \mathbf{z}', \mathbf{c} \rangle = \langle\mathbf{z}, \mathbf{c} \rangle \oplus \langle \mathbf{z}', \mathbf{c} \rangle = 0\oplus0=0 \Rightarrow \mathbf{z} \oplus\mathbf{z}' \in S_{\mathbf{c}}
     \end{equation}
    
     \item \textbf{Elemento neutro}: Veamos que $\mathbf{0}$ es elemento neutro, ya se probó en $S_{\mathbf{c}} \neq \varnothing$ que $\mathbf{0} \in S_{\mathbf{c}}$, $ \forall \:\mathbf{c} \in \{0,1\}^{n}$. Dada $\mathbf{z} \in S_{\mathbf{c}}$, $\mathbf{z} \oplus \mathbf{0} = \mathbf{z} = \mathbf{0} \oplus \mathbf{z}$, se verifica $\forall \:\mathbf{z} \in \{0,1\}^{n}$.

     \item \textbf{Elemento inverso}: Queremos ver que si $\mathbf{z} \in S_{\mathbf{c}}$ existe $\mathbf{z}^{-1} \in S_{\mathbf{c}}$ tal que $\mathbf{z}\oplus\mathbf{z}^{-1}=\mathbf{0}$. Por las propiedades de la suma bit a bit, sabemos que $\mathbf{z}$ es su propio inverso, es decir $\mathbf{z}^{-1}=\mathbf{z}$, debido a que para cada posición de la cadena ambos son $0$ o $1$ y la suma del bit, siempre es $0$. Por lo que, si $\mathbf{z} \in S_{\mathbf{c}} \Rightarrow \mathbf{z}^{-1} = \mathbf{z} \in S_{\mathbf{c}}$ tal que $\mathbf{z}\oplus\mathbf{z}^{-1}=\mathbf{z}\oplus\mathbf{z}=\mathbf{0}$.

     \item \textbf{Asociatividad}: Sean $\mathbf{z}_{1}, \mathbf{z}_{2}, \mathbf{z}_{3} \in S_{\mathbf{c}}$, $(\mathbf{z}_{1}\oplus \mathbf{z}_{2}) \oplus \mathbf{z}_{3} = \mathbf{z}_{1} \oplus( \mathbf{z}_{2} \oplus \mathbf{z}_{3})$ al ser $\oplus$ asociativa entonces se cumple la igualdad anterior.
 \end{itemize}

 Por lo que ya hemos probado que $\forall\;\mathbf{c}\in \{0,1\}^{n}$, $(S_{\mathbf{c}},\oplus)$ es un grupo.\newline

 \textbf{Implementación}: Para la implementación de esta regla hemos creado dos circuitos separados, una vez elegido un $\mathbf{c}\in \{0,1\}^{n}$, hemos creado el oráculo para esa cadena, nosotros hemos tomado $\mathbf{c}=|001\rangle$ para dibujar los circuitos y $\mathbf{c}=|0001\rangle$ para la obtención de soluciones, ya que queríamos evitar que el circuito fuera demasiado grande a la hora de mostrarlo.\newline

 El primer circuito, Figura \ref{Fig:CircuitoSimon1}, es el mismo que se presentó en el \hyperref[Sec3.5:Simon]{algoritmo de Simon}, donde obtenemos la cadenas $\mathbf{z}$. El segundo circuito, Figura \ref{Fig:CircuitoSimonR1}, lo que hacemos es aplicar en paralelo el algoritmo y sumar bit a bit ambos resultados antes de medir, de esta manera solo nos quedaría comprobar que los resultados son iguales. Que como se podrá observar en la Figura \ref{FIG:Simon1.resultado}, son idénticos al realizar ambas simulaciones.\newline

 
\textbf{Regla II}: Si simulamos la implementación del algoritmo a una cadena $\mathbf{c}$ y se invierten las cadenas $\mathbf{z}\in S_{\mathbf{c}}$, equivale a aplicar el programa a $\mathbf{c}$ invertida.\newline

Esto se debe a que si permutamos los bits para realizar la inversión tanto en el \textit{source input} como en el \textit{output}, no estamos variando el resultado. Se entiende de forma más sencilla con una perspectiva más matemática. Sean $M$ la matriz que realiza la inversión, no hay que olvidar que $M=M^{-1}$, y $\mathbf{z} \in S_{\mathbf{c}}$:

\begin{equation}
    \begin{split}
    \langle \mathbf{z},\mathbf{c} \rangle &= z_{1}c_{1} \oplus z_{2}c_{2} \oplus ...  \oplus z_{n}c_{n} \\ &= z_{n}c_{n} \oplus ... \oplus z_{2}c_{2} \oplus z_{1}c_{1} \\ &= \langle M\mathbf{z}, M\mathbf{c} \rangle = 0 \Rightarrow M\mathbf{z} \in S_{M\mathbf{c}}
    \end{split}
\end{equation}\newline

Esto se puede realizar de esta manera directa debido a que $\oplus$ es un operador conmutativo, por lo que incluso podríamos ampliar a que $(S_{\mathbf{c}},\oplus)$ es un grupo abeliano. Esto probaría la validez matemática de esta MR para el algoritmo de Simon. \newline

\textbf{Implementación}: simulamos el \hyperref[Fig:CircuitoSimon1]{algoritmo original} sobre $\mathbf{c}$ y $M\mathbf{c}$, invirtiendo una de las listas de cadenas obtenidas para comparar resultados. Estos pueden ser representados directamente en forma de listas, tal y como está en el documento o con la representación en histograma tras hacer la modificación.\newline

Todas estas simulaciones, así como el código para ambas reglas se pueden encontrar en el \href{https://github.com/rodelanu/TFG/tree/main}{repositorio común}, específicamente en el archivo sobre las reglas del algoritmo de Simon\footnote{\url{https://github.com/rodelanu/TFG/blob/main/3_Simon_Rules.ipynb}}.\newline

Para finalizar, habría que destacar que, tras realizar las pruebas implementadas, los resultado obtenidos han sido los esperados en todas las reglas propuestas en este texto.

\begin{figure}[H]
    \centering
    \includegraphics[width=0.9\textwidth]{TFG/imagenes/simonregla.png}
    \caption{Circuito, algoritmo Simon para la regla I con $\mathbf{c}=|001\rangle$}
    \label{Fig:CircuitoSimonR1}
 \end{figure}

 
 \begin{figure}[H]
    \centering
    \begin{subfigure}[H]{0.48\textwidth}
        \centering
        \includegraphics[width=\textwidth]{TFG/imagenes/resultadoSimon1.png}
        \caption{Algoritmo original}
        \label{sFig:Simon1.1}
    \end{subfigure}
    \hfill
    \begin{subfigure}[H]{0.48\textwidth}
        \centering
        \includegraphics[width=\textwidth]{TFG/imagenes/resultadoRegla1Simon.png}
        \caption{Regla I}
        \label{sFig:Simon1.2}
    \end{subfigure}
        \caption{Resultados de la simulación del algoritmo Simon en el circuito original y en el circuito con la regla I implementada, para la cadena $\mathbf{c}=|0001\rangle$. }
    \label{FIG:Simon1.resultado}
 \end{figure}